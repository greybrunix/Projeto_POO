\documentclass[10pt, a4paper]{article}
\usepackage{times}
\usepackage{titlesec}
\usepackage{graphicx}
\usepackage{lipsum}

\titlespacing*{\section}
{-1pt}{0.75\baselineskip}{.55\baselineskip}
\titlespacing*{\subsection}
{2pt}{.5\baselineskip}{.25\baselineskip}
\titlespacing*{\subsubsection}
{-4pt}{.3\baselineskip}{.1\baselineskip}
\title{Projeto POO\\
        Simulação de Casas Inteligentes}
\author{Bruno Gião a96544 \\ Miguel Vaz a72161 \\ João Cruz a95375 \\ Grupo 1}
\date{UM, LCC, 2021/2022}

\begin{document}
\maketitle

\newpage
\tableofcontents
\newpage

\section{Introdução}
        Neste projeto, temos como objetivo criar uma simulação de `Casas Inteligentes'.\@
        Ora, estas usam dispositivos inteligentes que têm certas regras que os definem. Regras essas que tornam este projeto ideal para um paradigma como
        o de Programação Orientada a Objetos.
        Através de encapsulamento, abstração, herança e composição, conseguimos realizar este projeto, assegurando a implementação devida de POO.\@
\section{Classes Base}
        De modo a implementar a tal `simulação' é necessário ter classes `base',
        nomeadamente as dos SmartDevices, SmartHouses e SmartEP.\@
\subsection{SmartDevices}
        SmartDevices são dispositivos que podem ser ligados ou desligados e têm um identificador único, no entanto, podem ser divididos em três tipos:
        SmartBulbs; SmartSpeakers e SmartCameras.
        Assim, o melhor modo de realizar esta tarefa foi claramente através do uso de uma classe Abstrata extendida pelas previamente mencionadas.
\subsection{SmartHouse}
        Uma SmartHouse é uma casa com dono e com uma coleção de SmartDevices, no caso deste projeto, essa coleção foi implementada através de um Map.
        De modo a garantir encapsulamento, estes são implementados através de composição.
\subsection{Fornecedor de Energia}
        Um fornecedor de energia é uma coleção de Casas, equipada com valores de impostos e de formulas para calcular preço por KWh.\@ 
        Fornecedores de energia também têm nome de empresa de modo a conseguir empresas armazenar num HashMap.
        De modo a colecionar estas casas, novamente, é utilizada composição.
\section{Simulador}
\lipsum[1]
\subsection{MVC}
\lipsum[1]
\subsubsection{Visualizador}
\lipsum[1]
\subsubsection{Modelo}
\lipsum[2]
\subsubsection{Controlador}
\lipsum[2]
\section{Diagrama UML}
Segue, a seguir o diagrama UML deste projeto:
\begin{figure}
        \centering
        \includegraphics[width=\textwidth]{diagram_1.png}
\end{figure}

\newpage
\section{Conclusão}
\lipsum[1-2]
\end{document}
